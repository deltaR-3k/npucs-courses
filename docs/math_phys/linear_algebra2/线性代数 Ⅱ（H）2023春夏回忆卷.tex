\documentclass[UTF8,14pt,normal]{ctexart}
\usepackage{amsmath}
\usepackage{amssymb}
\usepackage{geometry}
\geometry{a4paper,scale=0.66,top=0.8in,bottom=1.5in,left=1in,right=1in}

\title{\textbf{线性代数 II(H)2022-2023 春夏期末}}
\author{图灵回忆卷}
\date{2023 年 6 月 28 日}

\linespread{1.2}
\addtolength{\parskip}{.8em}

\begin{document}

\maketitle

\noindent{\heiti\textbf{一、(15 分)}} 已知 $ T \in \mathcal{L}(\mathbb{C}^3) $,其对应矩阵为
\[ \mathbf{A} = \begin{pmatrix} 0 & 0 & 0 \\ 2023 & 0 & 0 \\ 6 & 28 & 0 \end{pmatrix} \]

\textbf{1.} 求 $ \mathbf{A} $ 的 Jordan 标准形(不必求 Jordan 基);

\textbf{2.} 证明不存在复矩阵 $ \mathbf{B} $ 使得 $ \mathbf{B}^2 = \mathbf{A} $.

\noindent{\heiti\textbf{二、(17 分)}} 已知直线 $ L_1 = \begin{cases} x + y + z - 1 = 0 \\ x - 2y + 2 = 0 \end{cases} $,$ L_2 = \begin{cases} x = 2t \\ y = t + a \\ z = bt + 1 \end{cases} $,试确定 $ a $,$ b $ 满足的条件使得 $ L_1 $,$ L_2 $ 是:

\textbf{1.} 平行直线; \qquad \textbf{2.} 异面直线.

\noindent{\heiti\textbf{三、(18 分)}} 定义在 $ V = \mathbb{R}^3 $ 上的运算
\[ \langle \boldsymbol{x}, \boldsymbol{y} \rangle_V = x_1 y_1 + x_2 y_2 + (x_2 + x_3)(y_2 + y_3) \]
其中 $ \boldsymbol{x} = (x_1, x_2, x_3) $,$ \boldsymbol{y} = (y_1, y_2, y_3) $.

\textbf{1.} 验证 $ \langle \cdot, \cdot \rangle_V $ 是 $ \mathbb{R}^3 $ 上的一个内积;

\textbf{2.} 求 $ \mathbb{R}^3 $ 在 $ \langle \cdot, \cdot \rangle_V $ 下的一组标准正交基;

\textbf{3.} 求 $ \boldsymbol{\beta} \in V $ 使得 $ \forall \boldsymbol{x} \in V: x_1 + 2x_2 = \langle \boldsymbol{x}, \boldsymbol{\beta} \rangle_V $.

\noindent{\heiti\textbf{四、(15 分)}} $ T \in \mathcal{L}(V) $ 在一组基 $ \boldsymbol{\varepsilon} = (\varepsilon_1, \varepsilon_2, \varepsilon_3) $ 下的矩阵为
\[ T(\boldsymbol{\varepsilon}) = (\boldsymbol{\varepsilon}) \begin{pmatrix} 1 & 0 & 0 \\ 0 & 2 & 1 \\ 0 & 0 & 2 \end{pmatrix} \]
求 $ V $ 所有的 $ T $-不变子空间.

\noindent{\heiti\textbf{五、(20 分)}} 试给出下列命题的真伪. 若命题为真,请给出简要证明;若命题为假,请举出反例.

\textbf{1.} $ T \in \mathcal{L}(V) $. 若子空间 $ W \in V $ 在 $ T $ 下不变,则其补空间 $ W' $ 在 $ T $ 下也不变;

\textbf{2.} 定义 $ T \in \mathcal{L}(V, W) : Tv = \langle v, \alpha \rangle \beta,\ \beta \in W $ 对 $ \forall v \in V $ 成立,则 $ T^* w = \langle w, \beta \rangle \alpha,\ \alpha \in V $ 对 $ \forall w \in W $ 成立;

\textbf{3.} $ T \in \mathcal{L}(V) $ 是非幂零算子,满足 $ \operatorname{null} T^{n - 1} \neq \operatorname{null} T^{n - 2} $. 则其极小多项式为
\[ m(\lambda) = \lambda^{n-1}(\lambda - a) \qquad 0 \neq a \in \mathbb{R} \]

\textbf{4.} $ \mathbf{A} \in \mathbb{R}^{n \times n} $. $ \mathbf{S}_1 = \mathbf{A}^{\mathrm{T}} + \mathbf{A} $,$ \mathbf{S}_2 = \mathbf{A}^{\mathrm{T}} - \mathbf{A} $. 则 $ \mathbf{A} $ 是正规矩阵当且仅当 $ \mathbf{S}_1 \mathbf{S}_2 = \mathbf{S}_2 \mathbf{S}_1 $.

\textbf{5.} $ \mathbf{A} \in \mathbb{C}^{n \times n} $ 是正规矩阵,则 $ \mathbf{A} $ 的实部矩阵和虚部矩阵是对称矩阵.

\noindent{\heiti\textbf{六、(15 分)}} $ T \in \mathcal{L}(V) $. 有极分解 $ T = S \sqrt{G} $,其中 $ S $ 是等距同构,$ G = T^* T $. 证明以下条件等价:

\textbf{1.} $ T $ 是正规算子;

\textbf{2.} $ GS = SG $;

\textbf{3.} $ G $ 的所有特征空间 $ E(\lambda, G) $ 都是 $ S $-不变的.

\end{document}
