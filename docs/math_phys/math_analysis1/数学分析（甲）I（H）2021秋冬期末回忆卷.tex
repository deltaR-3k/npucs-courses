\documentclass[UTF8,14pt,normal]{ctexart}
\linespread{1.5}
\usepackage{amsmath}
\usepackage{geometry}
\geometry{a4paper,scale=0.66}

\title{\vspace{-5em}\bf 数学分析(甲)I(H)2021秋冬期末}
\author{图灵回忆卷}
\date{2022年1月5日}

\begin{document}
    \maketitle 
    
    \noindent{\heiti\textbf{一、(10分)}} 叙述Cauchy收敛原理,并证明数列$a_n=\displaystyle\sum_{k=1}^n\dfrac{\sin k}{k(k+1)}$收敛.
    
    \noindent{\heiti\textbf{二、(32分)}}计算\vspace{1em}
    
    \textbf{1.}$\displaystyle\lim_{x\to +\infty}\dfrac{\int_0^x(1+t^4)^\frac{1}{4}\mathrm dt}{x^3}$.\vspace{1em}
    
    \textbf{2.}$\displaystyle\int\dfrac{3x^2-4x-1}{x^3-2x^2-x+2}\mathrm dx$.\vspace{1em}
    
    \textbf{3.}$f(x)=\begin{cases}\dfrac{\sin x}{x}, &x\neq 0\\1, &x=0\\\end{cases}$,求$f'(0)$和$f''(0)$.\vspace{1em}
    
    \textbf{4.}求$y=e^x$过点$(0, 0)$的切线$L$,并求$y=e^x$、$L$和$y$轴所夹图形绕$x$轴旋转一周的体积.\vspace{1em}
    
    \noindent{\heiti\textbf{三、(8分)}}证明:无上界数列必存在发散于无穷大的子列.\vspace{0.5em}
    
    \noindent{\heiti\textbf{四、(10分)}}设$f(x)$在$[0, 1]$上定义,证明:\vspace{-0.5em}
    $$\displaystyle\sup_{x\in[0,1]}f(x)-\displaystyle\inf_{x\in[0,1]}f(x)=\displaystyle\sup_{x',x''\in[0,1]}\left|f(x')-f(x'')\right|.$$
    
    \noindent{\heiti\textbf{五、(10分)}}$f(x)$在$(0,1)$可导,证明:若$\displaystyle\lim_{x\to+\infty}f'(x)$存在,则$\displaystyle\lim_{x\to+\infty}f(x)$存在.\vspace{0.5em}
    
    \noindent{\heiti\textbf{六、(10分)}}证明:若$f(x)$在$[0,+\infty)$一致连续,$g(x)$在$[0,+\infty)$连续,且\\
    $\displaystyle\lim_{x\to+\infty}\left(f(x)-g(x)\right)=0$,则$g(x)$一致连续.\vspace{1em}
    
    \noindent{\heiti\textbf{七、(10分)}}$f(x)=\begin{cases}\dfrac{\sin x}{x}, &x\neq 0\\1, &x=0\\\end{cases}$,证明$\displaystyle\int_0^{+\infty}f(x)\mathrm dx$和$\displaystyle\int_0^{+\infty}f^2(x)\mathrm dx$均收敛,且其值相等.\vspace{0.5em}
    
    \noindent{\heiti\textbf{八、(10分)}}设$f(x)$三阶可导,$f(0)=f'(0)=f''(0)=0$,$f'''(0)>0$,$x\in(0,1)$时$f(x)\in(0,1)$,且数列$\{x_n\}$满足$x_{n+1}=x_n(1-f(x_n))$
    
    \textbf{1.}证明:$\displaystyle\lim_{n\to+\infty}x_n=0$;
    
    \textbf{2.}证明:存在$\alpha >0$和常数$c\neq 0$,使$\displaystyle\lim_{n\to+\infty}cn^\alpha x_n=1$.
    
    
    
\end{document}