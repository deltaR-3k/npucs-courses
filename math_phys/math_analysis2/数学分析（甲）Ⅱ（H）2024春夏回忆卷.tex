\documentclass[UTF8,14pt,normal]{ctexart}
\usepackage{amsmath}
\usepackage{physics} % \dd, \dv
\usepackage{mismath} % \ds
\usepackage{amssymb}
\usepackage{geometry}
\geometry{a4paper,scale=0.66,top=1in,bottom=1in,left=1in,right=1in}

\title{\vspace{-4em}\textbf{数学分析(甲)II(H)2023-2024 春夏期末}}
\author{图灵回忆卷}
\date{2024 年 6 月 20 日}

\linespread{1.1}
\addtolength{\parskip}{.2em}

\begin{document}

\maketitle

\noindent{\heiti\textbf{一、(10 分)}} 叙述二元函数 $f(x, y)$ 在 $(x_0, y_0)$ 可微的定义,并且证明以下函数在 $(0, 0)$ 处可微. \[f(x, y) = \begin{cases}
    y\arctan\dfrac{1}{\sqrt{x^2 + y^2}}  & (x, y) \neq (0, 0)\\
    0                                    & (x, y) = (0, 0)
\end{cases}\]

\noindent{\heiti\textbf{二、(32 分)}} 计算:

\textbf{1.} 求 $\ds\iiint\limits_{V}\sqrt{x^2 + y^2 + z^2}\dd{x}\dd{y}\dd{z}$,其中 $V = \{(x, y, z) \mid x^2 + y^2 + z^2 = 1, x\geqslant 0, y\geqslant 0, z\geqslant 0\}$;

\textbf{2.} 对于曲线 $L \colon y = \ds\int_0^x\sqrt{\sin t}\dd{t}\ (0\leqslant x\leqslant \pi)$,求 $\ds\int\limits_Lx\dd{s}$;

\textbf{3.} 对于曲线 $L \colon y = \sin x$,方向为从 $(0, 0)$ 到 $(\pi, 0)$,求 $\ds\int\limits_L(e^x\sin y-y^2)\dd{x} + e^x\cos{y} \dd{y}$;

\textbf{4.} 对于圆锥 $z = \sqrt{x^2 + y^2}\ (0\leqslant z\leqslant 1)$,方向为下侧,求 $\ds\iint\limits_{S}y^2\dd{z}\dd{x} + (z + 1)\dd{x}\dd{y}$.

\noindent{\heiti\textbf{三、(10 分)}} 设二元函数 $f(x, y)$ 在 $\mathbb{R}^2$ 上存在连续偏导数,且满足 $\ds\pdv{f}{x} + \pdv{f}{y} \neq 0$,$z$ 满足 $f(x - z, y - z) = 0$,证明:上式确定的隐函数 $z = z(x, y)$ 满足 \[\pdv{z}{x} + \pdv{z}{y} = 1.\]

\noindent{\heiti\textbf{四、(10 分)}} 利用条件极值证明 $(x_0, y_0, z_0)$ 到平面 $ax + by + cz + d = 0$ 的距离为 \[\rho = \dfrac{\lvert ax_0 + by_0 + cz_0 + d\rvert}{\sqrt{x_0^2 + y_0^2 + z_0^2}}.\]

\noindent{\heiti\textbf{五、(10 分)}} 叙述函数项级数 $\ds\sum\limits_{n=0}^{\infty} a_n(x)b_n(x)$ 一致收敛的 Dirichlet 判别法,并证明函数项级数 \[\sum_{n=1}^{\infty} \dfrac{n\cos(nx)}{n^2+1}\] 在 $(0, 2\pi)$ 内闭一致收敛.

\noindent{\heiti\textbf{六、(10 分)}} 求周期为 $2$ 的函数 \[f(x) = \begin{cases}
    x^2   & x\in [0 , 1) \\
    0     & x\in [-1, 0)
\end{cases}\] 的傅立叶展开,与该傅里叶级数在 $[-1, 1]$ 上的取值.

\noindent{\heiti\textbf{七、(10 分)}} 叙述常数项级数收敛的 Cauchy 准则并证明:若 $\ds\sum\limits_{n=1}^{\infty} a_n$ 收敛且 $\ds\sum\limits_{n=1}^{\infty}(b_{n+1} - b_n)$ 绝对收敛,则 $\ds\sum\limits_{n=1}^{\infty} a_nb_n$ 收敛.

\noindent{\heiti\textbf{八、(8 分)}} 设二元函数 $f(x, y)$ 在 $\mathbb{R}^2$ 存在二阶连续偏导数,对任意的 $\theta\in [0, 2\pi)$ 定义函数 \[g_\theta(t) = f(t\cos\theta, t\sin\theta).\]
若对于任意的 $\theta\in [0, 2\pi)$,$\ds\left.\dv{g_\theta}{t}\right\vert_{t=0} = 0$,$\ds\left.\dv[2]{g_\theta}{t}\right\vert_{t=0} > 0$,证明:$f(0, 0)$ 是 $f(x, y)$ 的极小值.

\end{document}
