\documentclass[UTF8,14pt,normal]{ctexart}
\linespread{1.5}
\usepackage{amsmath, amsfonts}
\usepackage{geometry}
\geometry{a4paper,scale=0.7,bottom=4cm,top=4cm}

\title{\vspace{-5em}\bf 数学分析(甲)II(H)2022春夏期末}
\author{21级图灵回忆卷}
\date{2022 年 6 月 15 日}

\begin{document}
    \maketitle 
    
    \noindent{\heiti\textbf{一、(10分)}} 叙述定义在区间$I$上的函数列$\{f_n\}$在$I$上一致收敛于$f(x)$的定义。并利用定义\vspace{0.6em}
    
    \noindent 证明$\left\{\dfrac{\sin(nx)}{n^2}\right\}$在$\mathbb{R}$上一致收敛.\vspace{0.5em}
    
    \noindent{\heiti\textbf{二、(10分)}}定义函数 $f(x, y)=\begin{cases}\dfrac{\sin(xy)}{\sqrt{x^2+y^2}}, &x^2+y^2\neq0\\0, &x^2+y^2=0\end{cases}$,证明$f(x, y)$在$(0, 0)$处连续且有偏导数,但在$(0, 0)$处不可微.\vspace{1em}
    
    \noindent{\heiti\textbf{三、(10分)}}利用依据说明$e^{x+y+1} -x^2y = e$可以确定唯一的隐函数$y=y(x)$,并求$\left.\dfrac{\mathrm dy}{\mathrm dx}\right|_{x=0}$和$\left.\dfrac{\mathrm d^2y}{\mathrm dx^2}\right|_{x=0}$.\vspace{1.2em}
    
    \noindent{\heiti\textbf{四、(32分)}}计算\vspace{1em}
     
    \textbf{1.}$\displaystyle\iiint_Vz^2\sqrt{x^2+y^2+z^2}\mathrm dx\mathrm dy\mathrm dz$,其中$V$为$\{(x, y, z)|x^2+y^2+z^2\leq R^2\}$,$R$为正常数.\vspace{0.5em}
    
    \textbf{2.}$\displaystyle\oint_L(z-y)\mathrm dx + (x-z)\mathrm dy + (x-y)\mathrm dz$,其中$L$为曲线$\begin{cases}x^2+y^2=1\\x-y+z=2\end{cases}$,方向为$z$轴正方向看为逆时针.\vspace{1em}
    
    \textbf{3.}$\displaystyle\int_Le^x(1-\cos y)\mathrm dx-e^x(1-\sin y)\mathrm dy$,其中$L$为$y=\sin x$从$(0, 0)$到$(\pi, 0)$的一段曲线.\vspace{1em}
    
    \textbf{4.}$\displaystyle\iint_\Sigma2xy\mathrm dy\mathrm dz+2yz\mathrm dx\mathrm dz+(z-2yz-z^2+1)\mathrm dx\mathrm dy$,其中$\Sigma$为上半球面$x^2+y^2+z^2=1, z\geq 0$,\vspace{-1em}
    
    \noindent 上侧为正侧.\vspace{1em}
    
    \noindent{\heiti\textbf{五、(10分)}}求函数$f(x, y)=xy+x-y$在$x^2+y^2\leq 5$上的最大值和最小值.\vspace{1em}
    
    \noindent{\heiti\textbf{六、(10分)}}求函数项级数$\displaystyle\sum_{n=0}^\infty\dfrac{x^n}{3^n(n+1)}$的收敛半径、收敛域以及和函数.\vspace{0.5em}
    
    \noindent{\heiti\textbf{七、(10分)}}设$f(x)$为周期为$2\pi$的周期函数,且$f(x) = \dfrac{1}{4}x(2\pi-x), 0\leq x\leq 2\pi$,将其展开为Fourier级数,并证明$\displaystyle\sum_{n=1}^\infty\dfrac{1}{n^2}=\dfrac{\pi^2}{6}$.
    
    \noindent{\heiti\textbf{八、(8分)}}设$f(x)$在$\mathbb{R}$上连续,定义函数列$f_n(x)=\dfrac{1}{n}\displaystyle\sum_{k=0}^{n-1}f\left(x+\dfrac{k}{n}\right)$,证明$f_n(x)$在$\mathbb{R}$上内闭一致收敛.
    
\end{document}